\section{Testing} % (fold)
\label{sec:testing}
Now that we have described how the work has been designed and then implemented,
it is time to ``prove'' that our solution works and leads to satisfying \emph{results}.
This section will thus describe how testing methods have been applied to our project
and how we managed to \emph{cover} most of the written code while testing.

While coding, each time a class (or group of classes working in the same pattern)
was completed, an associated \textsc{JUnit} file, containing diversified \lstinline|assert|
methods and \lstinline|@Before| statements to avoid repitition
(see listing~\ref{lst:test_total_income} to get an example),
was created and all important
methods had at least one test method.
Once we finished implementing the \Core~and the test associated
to it (it is logically the most detailed one), we had a total of 15 tests
files.
Even if test driven development wasn't our approach, we put a point
onto implementing the tests in detail and it seemed that they were
challenging all of our code.

At that point we began to wonder if it was possible to know \emph{exactly}
how much of the code was \emph{covered} by the tests.
Some research lead us to a very useful \textsc{Eclipse} plugin
called \textsc{EclEmma}~\cite{eclemma} that produces a very
detailed summary of the code coverage.
The first run of all tests using this plugin led to the
results shown in figure~\ref{fig:coverage_first}.
We observe a $80.2\%$ code coverage which seems intuitevely
fair enough as some of the code may not be useful to test.

\begin{figure}[h]
  \begin{center}
    \includegraphics[scale=0.47]{./img/coverage_start.png} 
  \end{center}
  \caption{Code coverage results when using \textsc{EclEmma} for
  the first time.}
  \label{fig:coverage_first}
\end{figure} 

Next comes the question of what code coverage percentage
we should achieve for it to be efficient.
It seems as ``\textit{the amount of testing necessary depends on a number of
factors}''~\cite{artimaHowMuchCoverage} that themselves depend
on how the code is written and what functions it implements.
Therefore, eventhough no particular goal for the coverage is set,
it allows to point out what part of the code \emph{might have been not tested}
or \emph{might be useless} and indeed led to $84.9\%$ after review (fig.~\ref{fig:coverage_end}).

\begin{figure}[h]
  \begin{center}
    \includegraphics[scale=0.47]{./img/coverage_end.png} 
  \end{center}
  \caption{Code coverage results after review of
  tests using first results (fig.~\ref{fig:coverage_first}).}
  \label{fig:coverage_end}
\end{figure}

% section testing (end)