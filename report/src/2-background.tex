\section{Background and requirements} % (fold)
\label{sec:background}

As already mentioned in the introduction, the project deals with the realisation of
a tool that simulates today's food delivery systems. The projet requires to have five main 
actors being the users: \Restaurant, \Customer, \Manager, \Courier~and the system (ie. the \Core).

A restaurant offers different \Meal, being either a \HalfMeal~or a  
\FullMeal. A \FullMeal~ is made of a \Starter, a \MainDish~and a \Dessert.
The \HalfMeal~consists of a \MainDish~and either a \Starter~or a \Dessert. 
Additionally each restaurant offers a \Menu~containing 
a list of \Starter, \MainDish~and \Dessert, so that a \Meal~can be chosen ``à la carte''. 

A \Customer~can choose one of three different \emph{fidelity card plans}
that allow him to some kind of reduction 
according to chosen \emph{fidelity card plan}.
All users contain basic attributes like a name, a surname, etc.
and have distinctive functions that are described 
in detail below in the ~\ref{sub:users}.
Moreover, the system implements different \emph{target profit policies},
\emph{delivery policies}, \emph{shipped order policies}
also being described in detail below \ref{sub:policies}.

Let's take a more detailed look at the requirements
needed for the core, the different actors,
the fidelity card plan and the supported policies.

\subsection{Core system} % (fold)
\label{sub:core_system}
\vspace{0.3\baselineskip}

\begin{itemize}
  \begin{minipage}{0.51\linewidth}
    \item setting of the service-fee
    \item setting of the markup percentage
    \item setting of the delivery cost
    \item notifying customers
  \end{minipage}
  \begin{minipage}{0.57\linewidth}
   \item allocating couriers to orders
   \item computing total income 
   \item choosing target profit policy
  \end{minipage}
\end{itemize}

% subsection core_system (end)

\subsection{Users} % (fold)
\label{sub:users}
\paragraph{Manager}~\vspace{0.3\baselineskip}
\begin{itemize}
  \begin{minipage}{0.47\linewidth}
    \item add/remove \User
    \item activate/deactivate \User
    \item profit related attribute
    \item income over period
    \item income per \Customer
  \end{minipage}
  \begin{minipage}{0.53\linewidth}
    \item target profit policy
    \item most selling \Restaurant
    \item most active \Courier
    \item setting delivery policy
  \end{minipage}
\end{itemize}

\paragraph*{Restaurant}~\vspace{0.3\baselineskip}
\begin{itemize}
  \begin{minipage}{0.47\linewidth}
    \item add/remove \Dish
    \item add/remove \Meal
    \item add/remove \Meal~of the week
  \end{minipage}
  \begin{minipage}{0.53\linewidth}
    \item set \lstinline|discountFactors|
    \item sort of shipped orders following policy
  \end{minipage}
\end{itemize}

\paragraph*{Customers}~\vspace{0.3\baselineskip}
\begin{itemize}
  \begin{minipage}{0.47\linewidth}
    \item place \Order
    \item choose fidelity plan option
    \item get history of orders
  \end{minipage}
  \begin{minipage}{0.53\linewidth}
    \item get info about fidelity plan and points
    \item access and modify account info
    \item give/remove consensus for notification
  \end{minipage}
\end{itemize}

\paragraph*{Couriers}~\vspace{0.3\baselineskip}
\begin{itemize}
  \begin{minipage}{0.47\linewidth}
    \item register/unregister their account
    \item set their state
  \end{minipage}
  \begin{minipage}{0.53\linewidth}
    \item change their position
    \item accept/refuse to a delivery call
  \end{minipage}
\end{itemize}

% subsection users (end)

\subsection{Policies} % (fold)
\label{sub:policies}
\paragraph{Target profit policies}~\vspace{0.3\baselineskip}
Two out of the three profit related quantities: \emph{service-fee,
the markup percentage, the delivery cost}
are taken as given inputs depending on the chosen \emph{target policy} as well as an expected profit.
The functions implemented by each policy allow the \Manager~to simulate the third 
profit related quantity that is not taken as an input and thus 
needed to achieve the profit. The profit is calculated as follows: 
\begin{center} 
  profit for one order = order price $\cdot$ markup percentage + service fee - delivery cost
\end{center}

\begin{itemize}
    \item \emph{targetProfit DeliveryCost}: the delivery cost is simulated.
    \item \emph{targetProfit ServiceFee}: the service fee is simulated.
    \item \emph{targetProfit Markup}: the markup percentage is simulated.
\end{itemize}

\paragraph{Delivery policies}~\vspace{0.3\baselineskip}
The delivery policies define in which way the \Core~choses the \Courier~for a placed \Order.

\begin{itemize}
    \item \emph{fastest delivery}: The courier having the shortest distance to the selected restaurant is chosen.
    \item \emph{fair-occupation delivery}: The courier having executed the least amount of orders is chosen.
\end{itemize}

\paragraph{Shipped order sorting policies}~\vspace{0.3\baselineskip}
The \emph{shipped order sorting policies} allow restaurants and managers
to sort the shipped orders according to different criterias.

\begin{itemize}
    \item \emph{most/least ordered Meal}: A ascending/descending list of all sold Meals
    \item \emph{most/least ordered dishes}: A ascending/descending list of all sold Dishes.
\end{itemize}

\paragraph{Fidelity Card Plan} 

Each \Customer can choose between three different fidelity card plans having each different bonuses.

\begin{itemize}
  \item \emph{Basic fidelity card}: The \Customer has access to a special offer provided by each restaurant.
  \item \emph{Point fidelity card}: The \Customer collects fidelity points
  and gets a certain reduction after 
	having reached a certain limit of points.
	\item \emph{Lottery fidelity card}: The \Customer takes part in a lottery
  and has the chance of getting his order for free.
\end{itemize}

% subsection policies (end)

\subsection{Summary} % (fold)

In order to give a basic overview of the typical function, the ordering of a food item,
we want to briefly summarize this process.
A \Customer~will place an \Order~by choosing a one or more \Meal~or one or more \Dish~of a \Restaurant.
The \Order~containing all the necessary information is handed to the \Core.
The \Core~will treat the \Order~by choosing a \Courier~according to the \emph{Delivery policy}. 
The \Courier~can either accept or decline the \Order. If any \Courier~accepts the \Order,
it is executed and saved in the \Core.

% subsection summary (end)

% section background (end)
