\section{Implementation} % (fold)
\label{sec:implementation}

In order to effectively give an overview of the code implementation
and to describe the key points, we will take the following approach: 

\begin{enumerate}
	\item First, we will describe the general structure of the
  main \emph{packages} (see~\ref{sub:structure_of_the_main_packages}).
	\item Second, we will pick out \emph{key methods} of the packages
  and explain them in detail (see~\ref{sub:key_methods}). 
	\item Third, we will describe the most important functionality of the system :
  \emph{placing an order} and treating it afterwards (see~\ref{sub:place_and_treat_order}).
\end{enumerate}

\subsection{Structure of the main packages} % (fold)
\label{sub:structure_of_the_main_packages}
Our java project contains in total nine packages.
The \exceptions~package will not be further explained as it is not too important.
The \parsers~and \txtF~ ones are there to create objects for the testing
and also need no further explanation.
The \userI~package is not relevant for the first part of the project
and will be described in the second part.
Lastly, the \tests~package will be explained in detail in the next section.
We will thus describe the \policies, \restaurantSetup, \policies, \tests~and
\users~packages. 

\subsubsection{Package core} % (fold)
\label{ssub:core}
The package \core~contains the classes \Core~and \Order~as can be seen in
the UML diagram in figure~\ref{fig:core_order_uml}. 

The core is the center of the application that \emph{connects all the users}.
To better understand the structure, please take a quick look at the UML diagram
in figure~\ref{fig:core_users_uml}. 
All classes that we made are there to give a certain functionality to a \User.
Since the \Core~contains a list of all \User, it is able to access all classes
and thus all the information that is saved in the system.
This characteristic is probably the most important one of our application since it allows us to

\begin{enumerate}
	\item handle all relevant functionalites over the \Core,
	\item ensure the open-close principle because we can easily decide
  which \User~has access to which functionality,
	\item and finally to take care that the structure is clear and well-organized.
\end{enumerate}

The \Order~class is there to present an actual made order.
An order is in some sense the way the different users ``communicate'' with restaurants.
Its most important functionalities will also be discussed throughoutly 
furter below (see~\ref{sub:place_and_treat_order}).
It is important to understand the different price attributes that are in 
the state of \Order as can be seen in listing~\ref{lst:prices_order}.

\begin{enumerate}
	\item \lstinline|priceInter| is the part of the order price the restaurant will receive.
	\item \lstinline|priceFinal| is the actual price the customer pays \MyFoodora.
	\item \lstinline|profitFinal| is the profit \MyFoodora~makes with the order. 
\end{enumerate}

% subsubsection core (end)

\subsubsection{Package policies} % (fold)
\label{ssub:policies}

The package \policies~contains all the necessary policies that are needed
and whose designs were explained in the previous section.

Most of the policies follow the typical \emph{strategy pattern},
where an interface with the name of the actual policy is implemented by the corresponding
behaviours classes, as shown in the UML diagram in figure~\ref{fig:delivery_uml}.
The core then has an attribute where its declaration is the class name
and which then is a reference to one of the actual policies (see 
figure~\ref{fig:core_policies_uml}). 
Only the \lstinline|SortPolicy|~is not realised according to
the typical \emph{strategy pattern}
and uses an \emph{abstract class} instead of an interface.
The following paragraph explains why we choose to do it this way
and what advantages it brings to the solution.

\textit{At this point it has to be mentioned that we well understood the assignement
to only display \HalfMeal, but we are convinced that is more logical
and user friendly to present all \Meal~instead of only halfmeals.}

We thought about using the \lstinline|ArrayList<savedOrders>|
of the core that saves all orders ever having been executed 
to present how often a \Meal~or \Dish~was sold, but we decided against it since
\begin{enumerate}
	\item First, it is not easy to extract the needed information
  from \lstinline|savedOrders|~because it contains objects of type \Order.
	\item Second, a new \lstinline|List|~would have to be created
  that has to sort all the needed information.
\end{enumerate}
Therefore, we implemented two sorted \lstinline|TreeSet|, 
ie. a list that is \emph{always sorted} because it puts the added 
element directly at the right spot. 
The advantages are
\begin{enumerate}
	\item Each time the function is called there is no sorting needed in advance.
	\item A lot of coding was avoided and \emph{transparency} is improved.
\end{enumerate}

% subsubsection policies (end)

\subsubsection{Package restaurantSetUp} % (fold)
\label{ssub:restaurantsetup}

The package \restaurantSetup~contains all the classes that are needed
for a restaurant to function properly.
Its corresponding UML diagram can be seen in figure~\ref{fig:dish_meal_uml}.
In order to add dishes or meals, we used the already explained \emph{abstract
factory pattern}. The two other important structures of 
\restaurantSetup~are the class structures of \Meal~and \Dish. 

\Meal~is inheritated by \HalfMeal~and \FullMeal. Since the function \lstinline|getPrice()|~thats add up the prices of the dishes included in the meal is the same for \FullMeal~and \HalfMeal, \Meal~does not have to be abstract. We also considered an extension of another type of \Meal, and were convinced that the functionality would still be the same so that a not abstract \Meal~is justified.
Because we do not want an object of type \Meal~to be created, the constructor is of type protected so that classes of other packages cannot access the constructor. We make sure that a \FullMeal~is composed of exactly one \Starter, one \MainDish~and one \Dessert~by declaring the parameters of the constructor respectively. The same holds true for \HalfMeal~taking the respective requirements into account.

\Dish~is inheritated by \Starter, \MainDish, \Dessert. All functionality for the concrete dishes is the same so the same reasoning as explained in the upper paragraph holds true. 

Lastly, it is to be noticed that \Restaurant~offers either a \Meal~or a self composed meal composed of different \Dish~of the menu. 
That is why we used \emph{composition} by giving each \Restaurant~exactly one \Menu~that contains all the single \Dish~being offered.
An object of type \Menu~contains therefore a list of \Starter, \MainDish, and \Dessert. Additionally, is a list of all the provided \Meal included in each \Restaurant.

An exemple will be explained below \ref{???}.

% subsubsection restaurantsetup (end)

\subsubsection{Package users} % (fold)
\label{ssub:users}

The package \users~contains the all the users of the system being: ``\textit{the carrier, the customer, the restaurant and the manager}''.
Each of them is inherited by \User~and therefore takes over certain necessary attributes like \lstinline|name|~and \lstinline|messageBox|~which is a \lstinline|Stack|~that contains all the messages intented for this \User. Additionally it is important that every \User~has the function \lstinline|update()|~that adds a message to the \lstinline|messageBox|, which is the reason why this method is also in the class \User. Each \User~adds specific functions for its needed functionality. We want to mention two important implementations:

\begin{enumerate}
	\item \Courier~has a \lstinline|LinkedList| of received orders, where an \Order~is placed every time he is choosen.
	\item \Customer~implements the interface \lstinline|Observer|~that is implemented by us (see  \ref{fig:observer_pattern_for}).
\end{enumerate}

\Manager~neither has many attributes, nor does it has many methods since all the functions used by \Manager~are in the \Core.
\Restaurant~is already sufficiently explained above \ref{ssub:restaurantsetup}.

% subsubsection users (end)

% subsection structure_of_the_main_packages (end)
\subsection{Key methods} % (fold)
\label{sub:key_methods}


\subsubsection{Log-in functionality of the core} % (fold)
\label{ssub:log_in_functionality_of_the_core}

The \emph{log-in/log-out functionality} of the core is very important because it ensures that certain function can only be accessed by certain \User. Let's take a look at the method \lstinline|logIn()|: \ref{lst:login}. The input is a username. First the function checks whether the \lstinline|username|~is registered and only if this holds true the \User~having this \lstinline|username|~can log in. Then the \lstinline|current_user|~being an attribute of \Core~ and of type \User~is given the \User~that is associated to the \lstinline|username|~via the \lstinline|HashMap|~\lstinline|users|. Now, the function checks which type of \User~has just logged in sets the respective ``current" attribute to the \User~being now logged in. If for exemple a \Manager~logs in the attribute \lstinline|current_manager|~will be set to the \User, whereas the other ``current" attributes (respectively \lstinline|current_restaurant|, \lstinline|current_customer| and \lstinline|current_courrier|) stay equal to \lstinline|null|. Finally the left messages for the logged-in \User~are displayed via calling the method \lstinline|current_user.checkMessages()|.

This system allows us to design pretty much every method in the core in a way that it contains in the beginning a check who is actually logged-in, e.g.: \lstinline|if(current_manager != null){|. Herewith, we can ensure that certain functions can only be used by certain \User and that a log-in had to be performed before a \User~can use a method.

The logged-out function is self-explanatory: \ref{lst:logout}.

% subsubsection log_in_functionality_of_the_core (end)

\subsubsection{Fastest Delivery method of Delivery Policy} % (fold)
\label{ssub:fastest_delivery_method_of_delivery_policy}

The \emph{Fastest Delivery method} is presented at this point because it gives a good exemple of how the implementation of the \emph{Strategy pattern} actual works and also shows how this project allowed us to gain new skills in \lstinline|Java|. Please take a look at the following pseudo code in order to better understand the method that is talked about: 
\ref{lst:fast_deliv_meth}. 

To begin with, it can easily seen that the method is the method that has to be overridden due to the implementation of the interface \lstinline|DeliveryPolicy|. Every object that is declared as an \lstinline|DeliveryPolicy|~object thus has to have a function called \lstinline|howToDeliver|. So what does this function do? In short, this method takes a list of all couriers and an address (of a \Restaurant) as entries and returns a list of the same couriers sorted in ascending order according to the distance between the couriers address and the inserted restaurant's address. Let's dive into the code.

First the method calls another method called \lstinline|getDistance|~that takes the same entries as the \lstinline|howToDeliver|~method and gives back an \lstinline|ArrayList|~of type \lstinline|double|~containing the distances between each courier and the restaurant's address. The list is unsorted and looks for exemple as follows: \lstinline|{4.22, 8.9, 9.34, 8.3, ...}|. This method performs a simple calculation and will not be explained it detail.

Second comes the actual sorting part of the function. In the beginning, we did not find an efficient way to sort the couriers of list by a value that is not even an attribute of the object \Courier. After searching the internet, we discovered the \lstinline|Stream|~functionality of \lstinline|Java|~and after learning the basics of this Java8 extension saw in it a good solution to our problem. So, we initialise a \lstinline|Stream|~having the amount of members as there are couriers in the list and map each of those members to a new Object, called 
\lstinline|CourierDistance| that contains a \Courier~as well as a distance. We created this class to actually sort the \Courier~by their respective distance, being an external value that changes depending on the restaurants \lstinline|Address|. Now our stream is basically an unsorted list of objects of type \lstinline|CourierDistance|. The next step is to sort the list members according to their distances. Finally, the objects \lstinline|CourierDistance|~are mapped to the type \Courier~ and the \lstinline|Stream|~is transformed into an \lstinline|ArrayList|~that will be returned and used to chose the best fit courier for a specific order. 

The background of this method will be clearer after having read the third part of this section \ref{???}. This code excerpt should have shown:

\begin{enumerate}
	\item how the \emph{Strategy pattern} works in practise and
	\item how we always looked for efficient code designing.
\end{enumerate}

% subsubsection fastest_delivery_method_of_delivery_policy (end)

\subsubsection{Add Full Meal method} % (fold)
\label{ssub:add_full_meal_method}



% subsubsection add_full_meal_method (end)

% subsection key_methods (end)








\subsection{Place and treat order} % (fold)
\label{sub:place_and_treat_order}

hello

% subsection place_and_treat_order (end)


% section implementation (end)