\section{Implementation} % (fold)
\label{sec:implementation}

In order to effectively give an overview of the code implementation
and to describe the key points, we will take the following approach: 

\begin{enumerate}
	\item First, we will describe the general structure of the
  main \emph{packages} (see~\ref{sub:structure_of_the_main_packages}).
	\item Second, we will pick out \emph{key methods} of the packages
  and explain them in detail (see~\ref{sub:key_methods}). 
	\item Third, we will describe the most important functionality of the system :
  \emph{placing an order} and treating it afterwards (see~\ref{ssub:process_of_giving_up_an_order}).
\end{enumerate}

\subsection{Structure of the main packages} % (fold)
\label{sub:structure_of_the_main_packages}
Our java project contains in total nine packages.
The \exceptions~package will not be further explained as it is not too important.
The \parsers~and \txtF~ ones are there to create objects for the testing
and also need no further explanation.
The \userI~package is not relevant for the first part of the project
and will be described in the second part.
Lastly, the \tests~package will be explained in detail in the next section.
We will thus describe the \policies, \restaurantSetup, \policies, \tests~and
\users~packages. 

\subsubsection{Package core} % (fold)
\label{ssub:core}
The package \core~contains the classes \Core~and \Order~as can be seen in
the UML diagram in figure~\ref{fig:core_order_uml}. 

The core is the center of the application that \emph{connects all the users}.
To better understand the structure, please take a quick look at the UML diagram
in figure~\ref{fig:core_users_uml}. 
All classes that we made are there to give a certain functionality to a \User.
Since the \Core~contains a list of all \User, it is able to access all classes
and thus all the information that is saved in the system.
This characteristic is probably the most important one of our application since it allows us to

\begin{enumerate}
	\item handle all relevant functionalites over the \Core,
	\item ensure the open-close principle because we can easily decide
  which \User~has access to which functionality,
	\item and finally to take care that the structure is clear and well-organized.
\end{enumerate}

The \Order~class is there to present an actual made order.
An order is in some sense the way the different users ``communicate'' with restaurants.
Its most important functionalities will also be discussed throughoutly 
furter below (see~\ref{ssub:process_of_giving_up_an_order}).
It is important to understand the different price attributes that are in 
the state of \Order as can be seen in listing~\ref{lst:prices_order}.

\begin{enumerate}
	\item \lstinline|priceInter| is the part of the order price the restaurant will receive.
	\item \lstinline|priceFinal| is the actual price the customer pays \MyFoodora.
	\item \lstinline|profitFinal| is the profit \MyFoodora~makes with the order. 
\end{enumerate}

% subsubsection core (end)

\subsubsection{Package policies} % (fold)
\label{ssub:policies}

The package \policies~contains all the necessary policies that are needed
and whose designs were explained in the previous section.

Most of the policies follow the typical \emph{strategy pattern},
where an interface with the name of the actual policy is implemented by the corresponding
behaviours classes, as shown in the UML diagram in figure~\ref{fig:delivery_uml}.
The core then has an attribute where its declaration is the class name
and which then is a reference to one of the actual policies (see 
figure~\ref{fig:core_policies_uml}). 
Only the \lstinline|SortPolicy|~is not realised according to
the typical \emph{strategy pattern}
and uses an \emph{abstract class} instead of an interface.
The following paragraph explains why we choose to do it this way
and what advantages it brings to the solution.

\textit{At this point it has to be mentioned that we well understood the assignement
to only display \HalfMeal, but we are convinced that is more logical
and user friendly to present all \Meal~instead of only halfmeals.}

We thought about using the \lstinline|ArrayList<savedOrders>|
of the core that saves all orders ever having been executed 
to present how often a \Meal~or \Dish~was sold, but we decided against it since
\begin{enumerate}
	\item First, it is not easy to extract the needed information
  from \lstinline|savedOrders|~because it contains objects of type \Order.
	\item Second, a new \lstinline|List|~would have to be created
  that has to sort all the needed information.
\end{enumerate}
Therefore, we implemented two sorted \lstinline|TreeSet|, 
ie. a list that is \emph{always sorted} because it puts the added 
element directly at the right spot. 
The advantages are
\begin{enumerate}
	\item Each time the function is called there is no sorting needed in advance.
	\item A lot of coding was avoided and \emph{transparency} is improved.
\end{enumerate}

% subsubsection policies (end)

\subsubsection{Package restaurantSetUp} % (fold)
\label{ssub:restaurantsetup}

The package \restaurantSetup~contains all the classes that are needed
for a restaurant to function properly.
Its corresponding UML diagram can be seen in figure~\ref{fig:dish_meal_uml}.
In order to add dishes or meals, we used the already explained \emph{abstract
factory pattern}. The two other important structures of 
\restaurantSetup~are the class structures of \Meal~and \Dish. 

\Meal~is inheritated by \HalfMeal~and \FullMeal. 
It is important to know that even if the class \Meal~does not have to be abstract,
we declared it as an abstract class either way. 
The reason is that, we want to prevent other classes being able to 
create an instance of this type, as this have no meaning for the solution.
We make sure that a \FullMeal~is composed of exactly one \Starter,
one \MainDish~and one \Dessert~by declaring the parameters of the method
\lstinline|setFullMeal| respectively.
The same holds true for \HalfMeal~taking the respective requirements into account.

On the other side, \Dish~is inheritated by \Starter, \MainDish~and \Dessert.
All functionalities for the concrete dishes are the same so the same 
reasoning as explained for meals holds true. 

Lastly, it is to be noticed that a \Restaurant~offers either a \Meal~or
a self composed meal composed of different dishes of the menu.
That is why we used \emph{composition} by giving each \Restaurant~exactly
one \Menu~that contains all the single dishes being offered.
An object of type \Menu~contains therefore a list of starters, maindishes and desserts.
Additionally, is a list of all the provided meals included in each \Restaurant.

% subsubsection restaurantsetup (end)

\subsubsection{Package users} % (fold)
\label{ssub:users}

The package \users~contains all classes related to users of the system,
being ``\textit{the courier, the customer, the restaurant and the manager}''.
Each of them is inherited by \User~and therefore takes over certain 
necessary attributes like \lstinline|name| and \lstinline|messageBox|
which is a \lstinline|Stack| that contains all the messages intented for this \User.
Note that (as for the \Dish~and \Meal~classes), the \User~class is declared \emph{abstract},
preventing ``unintented users'' instanciation.
Additionally it is important that every \User~has the function \lstinline|update()|
that adds a message to its \lstinline|messageBox|, which is the reason 
why this method is also in the class \User.
Each \User~adds specific functions for its needed functionality.
We want to mention two important implementations
\begin{enumerate}
	\item \Courier~has a \lstinline|LinkedList| of received orders,
  where an \Order~is placed every time he has to deliver it.
	\item \Customer~implements the interface \lstinline|Observer|,
  as shown in figure~\ref{fig:users_uml}.
\end{enumerate}

\Manager~neither has many attributes, nor does it has many methods 
since all the functions used by \Manager~are in the \Core.

% subsubsection users (end)

% subsection structure_of_the_main_packages (end)
\subsection{Key methods} % (fold)
\label{sub:key_methods}


\subsubsection{Log-in functionality of the core} % (fold)
\label{ssub:log_in_functionality_of_the_core}

The \emph{log-in/log-out functionality} of the core is very important because it ensures 
that certain functions can only be accessed by certain users.
Let's take a closer look at the method \lstinline|logIn()| (see listing~\ref{lst:login}).
First the function checks whether the \lstinline|username|~is
registered and only if this holds true the \User~having this \lstinline|username| can log in.
Then the \lstinline|current_user| being an attribute of \Core~and
of type \User~is given the \User~that is associated to the \lstinline|username|
via the \lstinline|users HashMap|.
Now, the function checks which type of \User~has just logged in sets 
the respective ``current" attribute to the \User~being now logged in.

If for exemple a \Manager~logs in the attribute \lstinline|current_manager| will
be set to the \User, whereas the other ``current" attributes (respectively
\lstinline|current_customer|, \lstinline|current_restaurant|
and \lstinline|current_courrier|) stay equal to \lstinline|null|.
Finally the left messages for the logged-in \User~are displayed
via calling the method \lstinline|checkMessages|.

This system allows us to design pretty much every method in the core in a defensive way 
by containing a ``checker'' verifying who is actually logged-in, e.g.
\begin{center}
  \lstinline|if(current_customer != null){ customerAction(); }|
\end{center}
Herewith, we can ensure that certain functions can only be used 
by certain \User~and that a log-in had to be performed before a \User~can use a method.

A point has to be made on the \emph{efficiency} of the login function, which
is somewhat important in such systems where the users don't want to wait
while logging in.
Indeed, the use of an \lstinline|HashMap| allows to check if username exists
in $\mathcal{O}(1)$ time on average and in $\mathcal{O}(\log{n})$ on worst case
(this happens when collisions happen, ie. the \lstinline|HashMap|
entries container is not big enough)\cite{hashMap}.

% subsubsection log_in_functionality_of_the_core (end)

\subsubsection{Fastest Delivery method of Delivery Policy} % (fold)
\label{ssub:fastest_delivery_method_of_delivery_policy}

The \emph{Fastest Delivery method} is presented at this point because it gives a good example
of how the implementation of the \emph{Strategy pattern} actually works
and also shows how this project allowed us to gain new skills in \textsc{Java}.
Please take a look at the following pseudocode in listing~\ref{lst:fast_deliv_meth}
in order to better understand the algorithm that will be described. 

To begin with, the method in question is the one that has to be overridden
due to the implementation of the \lstinline|DeliveryPolicy| interface.
Every object that is declared as a \lstinline|DeliveryPolicy|~object
and thus has to have a function called \lstinline|howToDeliver|. 
So what does this function do? In short, this method takes a list of all couriers
and an address (of a \Restaurant) as input and returns a list of the same couriers
sorted in ascending order according to the \emph{distance} between the couriers
address and the inserted restaurant's address. Let's dive into the code.

First the method calls another method called \lstinline|getDistance|~that
takes the same entries as the \lstinline|howToDeliver|~method and updates the
\lstinline|ArrayList<Double>|~containing the distances, computed using the
classical 2D Euclidean frame distance,
\[
  d = \sqrt{(x_2 - x_1)^2 + (y_2 - y_1)^2}
\]
between each courier and the restaurant's address. 
The list is unsorted and looks for example as follows: \lstinline|{4.22, 8.9, 9.34, 8.3}|. 

Second comes the actual sorting part of the function. In the beginning,
we did not find an efficient way to sort the couriers list by a value that is not
even an attribute of the object \Courier.
After searching the internet, we discovered the \lstinline|Stream|~functionality
and after learning the basics of this \textsc{Java 8} extension, we saw in it a good solution
to our problem. So, we initialise a \lstinline|Stream| having the amount of members as
there are couriers in the list and map each of those members to a new Object,
called \lstinline|CourierDistance| that contains a \Courier~as well as a distance.
We created this class to actually sort the \Courier~by their respective distance,
being an external value that changes depending on the restaurants \lstinline|Address|.
Now our stream is basically an unsorted list of objects of type \lstinline|CourierDistance|.

The next step is to sort the list members according to their distances,
Finally, the objects \lstinline|CourierDistance|~are mapped to the type \Courier~and
the \lstinline|Stream| is transformed into an \lstinline|ArrayList| 
that will be returned and used to choose the best fit courier for a specific order. 

The background of this method will be clearer after having read 
the third part of this section (see \ref{ssub:process_of_giving_up_an_order}
on giving up an order).

% subsubsection fastest_delivery_method_of_delivery_policy (end)

\subsubsection{Add \FullMeal~method} % (fold)
\label{ssub:add_full_meal_method}

This section explains how a restaurant will actually add a \FullMeal.
In order to create a fullmeal, one creates a new \lstinline|MealFactory| as can been seen in
listing~\ref{lst:createFactory}. Calling the static \lstinline|getFactory()|
method with the parameter ``\textsc{meal}" returns a
\lstinline|MealFactory|. Having created the factory, the next step is to create 
the actual \FullMeal. The just created \lstinline|MealFactory| now enables one to
create an actual \Meal~as can be seen in listing \ref{lst:mealCreator}. Calling the
method \lstinline|getMeal()| with the name of the \Meal~and the parameter ``\textsc{fullmeal}"
finally returns a \FullMeal~that does not yet include the actual dishes belonging to
the meal. 

To complete the \FullMeal, the method \lstinline|setFullMeal()| with the three
parameters of type \Dish~ being notably \Starter, \MainDish~and \Dessert, is called,
which can be seen in listing~\ref{lst:SetFullMeal}. First, the three dishes are added to
the \FullMeal. Second, the method automatically recognizes whether all added dishes are
of the same type by a \lstinline|type1.equals(type2)| comparison and names the type of the
\Meal~accordingly. The \FullMeal~is finally completed and can be added to \Restaurant. 
% The self-explanatory method of the class \Restaurant~can be seen in listing \ref{lst:addMeal}.

The question arises, whether it is really necessary to call that many methods only to 
add a \FullMeal~to a \Restaurant. This brings us to the disadvantage of
implementing the \emph{abstract factory pattern}. Having assessed the different
designing possibilities, we considered the easy-to-extend design of the 
\emph{abstract factory pattern} to be the best fit since a likely extension 
of the restaurant's offer (e.g. drinks) can very easily implemented. 
The slightly more complex process of creating a \FullMeal will stay unnoticed
to the actual user because he will be guided by the system while inserting
the needed information for a \FullMeal.

% subsubsection add_full_meal_method (end)

% subsection key_methods (end)

\subsection{Process of giving up an order} % (fold)
\label{ssub:process_of_giving_up_an_order}

In the last part of the implementation section, we quickly want to go through
important functions being \emph{placing} an order and then \emph{treating} it. 
So, making the hypothesis that a customer created an \Order~by having chosen a
\Restaurant~and having added all the necessary meals or dishes,
we launch the \lstinline|placeOrder()| method with the parameter \Order~as can be
seen in listing~\ref{lst:placeOrder}. 
The first lines of the code ensure, as we explained earlier, that a \Customer~is
actually logged in to give up an \Order. The created order will be saved in the
\lstinline|LinkedList| of \lstinline|receivedOrders| of the \Core. The use of a
\lstinline|LinkedList| realises the so-called ``First-in-First-out",
which is reasonable considering the fact that the oldest \Order~should be treated first.
Next, the \Customer~will be notified and the order can be treated.

Now, we arrive at the main method of the system the \lstinline|treatOrder()| method.
Please take a look at both the simplified pseudocode that will be
clarified in the listing~\ref{lst:treatOrder}.

The function is declared \lstinline|private| so that it can only be
called in the core and this is done solely by the function \lstinline|placeOrder()|.
This means that as soon as a logged-in \Customer~places an \Order~it will be taken
care of instantely, so that in the current version of our system each \Order~that is
placed in the list of \lstinline|receivedOrders| is removed and treated on the spot.
The system, extended to a multi-threaded one, takes advantage of the implementation of
a list of \lstinline|receivedOrders|, whereas it is useless for the system as it is now. 

\begin{itemize}
	\item First, the oldest order is taken and will be treated. 
	\item Second, a list of couriers is generated according to the current \lstinline|DeliveryPolicy|.
	\item Third, now that we have a list of courier we will contact them one by one
	according to the order of the courier list until either a \Courier~accepted
	the \Order~or there are currently no \Courier~available. Then this \Courier~can either accept
	or decline the \Order. Since the system is a single-threaded system, we use a
	function that \emph{randomly} makes the courier accept or decline the \Order
  (we choose to use a $90\%$ acceptance rate). Finally, the \Courier~is removed from the list.
	\item Forth, if all couriers delined, the system will notify the \Customer
  that the \Order~will be deleted and the method ends.
	\item Fifth, if a \Courier~was found, the final price of the order will be saved
	as an attribute of the \Order. Lastely the \Order~is saved and all related
	users will be notified.	
\end{itemize}

% subsection place_and_treat_order (end)

We have now detailed the relevant points of our implementation and the next section
will describe the testing techniques that we used.

% section implementation (end)