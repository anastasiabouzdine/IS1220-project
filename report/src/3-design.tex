\section{Analysis and design} % (fold)
\label{sec:analysis_and_design}
This section will describe the design of our solution and how
it meets the given requirements presented in section~\ref{sec:background}.
The first part will present the different \emph{design patterns}
used throughout the code.
These patterns are useful as they allow
\begin{enumerate}
  \item to avoid redesigning,
  \item to follow guidelines so that the solution is extendable,
  \item to isolate parts of the code that might undergo changes.
\end{enumerate}

\subsection{Factory pattern for \texttt{Dish} and \texttt{Meal} classes} % (fold)
\label{sub:factory_pattern_for_the_texttt_dish_and_texttt_meal_classes}
When going through the different requirements for the two classes, a first thought
might be to implement two factory patterns, one for each.
This lead us to the first type of factory pattern known as \emph{parametric factory pattern}.
But the latter is not easily extendable to new features, such as
a \Drink~class to handle all types of drinks, especially if
those belong to different families (as \Dish, \Meal~and \Drink).
Therefore, the \emph{abstract factory pattern} seems to be the
most appropriate choice to respect the open/close principle.

This type of factory pattern allows us to seperate the \Dish~and \Meal~creation from their own functions, 
as the constructor is outsourced to another class - the factory.
The listing~\ref{lst:factory} illustrates how it can be applied and why it is powerful.
We have one \lstinline|AbstractFactory| object that is able to produce
two types of dishes (ie. a \Starter~and a \MainDish).

\begin{lstlisting}[caption=Factory pattern for \Dish~and \Meal.,
label=lst:factory]
AbstractFactory dishF = FactoryProducer.getFactory("Dish");

Dish d1 = dishF.getDish("starter", "avocado", 6.5);
d1.setType("vegetarian");

Dish d2 = dishF.getDish("maindish", "ham with salad", 12.5);

double price = d1.getPrice() + d2.getPrice();
\end{lstlisting}

% subsection factory_pattern_for_the_texttt_dish_and_texttt_meal_classes (end)

\subsection{Observer pattern for \texttt{Customer} and \texttt{Core} class} % (fold)
\label{sub:observer_pattern_for}
The second use of a design pattern comes into place when implementing the function that
a restaurant can \emph{notify the customers} when it changed the special meal of the week offer. 
This can be compared to different \emph{subscribers} being interested in updates of a \emph{publisher}. 
The relation with the project is described in table~\ref{tab:observer}.

It is important to note here that the publisher is not the \Restaurant.
This might seem counter-intuitive at first sight but in our case 
it is the \Core~object that
\begin{itemize}
  \item keeps track of the registered customers,
  \item notifies those who gave their consensus,
  \item knows when a \Restaurant~changes his meal of the week.
\end{itemize}

\begin{table}[H]
  \centering
  \begin{tabular}{|l|l|}
    \hline
    \textbf{Observer pattern} & \textbf{Applied to \MyFoodora}\\
    \hline
          Suscribers &             \Customer \\
          Publisher &              \Core\\
          Change of state &        updated \Meal~of the week\\
    \hline
  \end{tabular}
  \caption{Relationship between Observer pattern in theory and practical application for \MyFoodora.}
  \label{tab:observer}
\end{table}

The basic idea of the implementation of this pattern is given
in the listing~\ref{lst:observer} (in order not to overload this section with listings, most
of them have been placed to the appendix~\ref{app:code_listing} page~\pageref{app:code_listing}). 
We notice the use of a \lstinline|beNotified| boolean attribute by the \Customer.
The use of this attribute is a design choice that seems more
efficient than keeping a list of ``I accept to be notified users''
for two reasons.
First, we won't need to have another list in the \Core~containing
references to already used objects (ie. users), and second,
there is no adding or removing from a list needed when a \Customer~wants to
change its notification system, but only a primitive assignment.

% subsection observer_pattern_for (end)

\subsection{Strategy pattern for the \texttt{MyFoodora} policies} % (fold)
\label{sub:strategy_pattern_for_the_texttt_myfoodora_policies}
Once the classes and subclasses being related to the users and the 
restaurants had been implemented, the coding of the \Core~(ie. ``one to rule them all'') had to be started.
The first part consisted of designing the different policies that
the system supports. These policies can be seen as different \emph{behaviours}
of the system. Instead of using inheritance, which seems the
most intuitive approach, it is more efficient to use \emph{aggregation}.
In the project's context this means that the \Core~will have references
to objects of the \lstinline|Policies|~explained in section~\ref{sec:background}.
Application of this concept to our code can be found in the listing~\ref{lst:aggregationCore}.

The application of this pattern is useful as it allows to change from one policy to
another with a simple call of a function f.e. \lstinline|setTPolicyToFastestDelivery()|~ without having to change the \Core~class functions.
Also, adding a new option for a policy only requires a further
implementation of the policy interface.
The listing~\ref{lst:strategy} shows how we applied to strategy pattern
for the DeliveryPolicy and the table~\ref{tab:strategy}
links the theorical view of the pattern and the practical application the project.

\begin{table}[H]
  \centering
  \begin{tabular}{|l|l|}
    \hline
    \textbf{Strategy pattern} & \textbf{Applied to \MyFoodora}\\
    \hline
          Context &             \Core  \\
          Strategy &             \lstinline|DeliveryPolicy| \\
          ConcreteStrategyA &     \lstinline|FastestDelivery| \\
          ConcreteStrategyB &     \lstinline|FairOccupationDelivery| \\
    \hline
  \end{tabular}
  \caption{Relationship between Strategy pattern in theory and practical application for \MyFoodora.}
  \label{tab:strategy}
\end{table}


% subsection strategy_pattern_for_the_texttt_myfoodora_policies (end)

\subsection{What about the Singleton pattern ?} % (fold)
\label{sub:what_about_the_singleton_pattern}
Towards the end of the project, while implementing the \Core,
we thought that it would be a good idea to limit
the instantionation of the \Core to only one single instance
since the client would only create one instance.
This reminded us of the \emph{Singleton pattern} seen in class.
We therefore listed all possible solutions on how to code
it efficiently and we came to the implementation described
in the listing~\ref{lst:singleton}.
The singleton uses the so called \emph{Holder} technique~\cite{goodSingleton},
which prevents two threads from creating two different singletons
while staying memory efficient as the internal class will only be
loaded once (ie. at the call of the \lstinline|getInstance()| method).

Implementing this pattern immediately led to many problems with
the \textsc{JUnit} tests because once the singleton was defined in the test class it
was impossible to reset the instance, a necessity that is done extensively
in the \Core test, the latter being the most important test file.
We thus reconsidered using this pattern.
Some research~\cite{singletonLiars} led us to believe that using a singleton for this kind
of project brought more harm than good for two reasons.
Indeed, the ``\textit{core of the issue is that the global instance 
variables have transitive property. All of the internal objects of the 
singleton are global as well (and the internals of those objects are 
global as well\dots recursively)}''\cite{codeHardToTest}.
Moreover, the problem of reusing the code and not knowing that
there can only be one single instance of the \Core class,
which always seems normal to the writer of the code but not
to somenone from the outside.
Last but not least, the fact that the second part of the project consists of
designing a \emph{user interface} allow us to ensure that there is only one instanciation
for every usage of the program.

\begin{lstlisting}[caption=How the implementation of the Singleton pattern
  would look like.,
  label=lst:singleton] 
public Core() {
}
/** Holder of the singleton */
private static final class CoreHolder {		
  private static final Core instance = new Core();
}
/** Getter of the unique singleton */
public static Core getInstance() {
  return CoreHolder.instance;
}
\end{lstlisting}
  
% subsection what_about_the_singleton_pattern (end)

% section analysis_and_design (end)