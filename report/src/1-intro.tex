\section{Introduction} % (fold)
\label{sec:introduction}
Nowadays, more and more software programs are needed and designed every day.
These play an important role in everyday life and more specifically
in domains like ``\textit{manufacturing, banking, travel, communications,
defense, medicine, research, government, education, entertainement,
law, etc.}''\cite[p.6]{long2008critical}.
The goal of this project is to develop a software solution,
called \lstinline|MyFoodora|, whose functions are similar to those
of today food delivery systems\footnote{Such as \url{https://www.foodora.fr/}
and \url{https://deliveroo.fr/}.}.

The aim of this project was not only to have a solution that \emph{worked},
meaning it gives the expected output for a given input,
but that was \emph{resusable} and easily \emph{extendable} to new functionnalities.
For this to be achieved, we apply \emph{design patterns} in order
to follow the well-known \emph{open/close principle}.
This is why we thourougly thought and discussed which patterns to follow
and implement before effectively beginning to code.
This brainstorming approach allowed us
\begin{enumerate}
  \item to keep a global overview of what was needed at which point and why,
  \item to acquire a good comprehension of the pros and cons of each
  design patterns.
\end{enumerate}

The implementation of the solution has not been without done without \emph{difficulty}.
Simply said, the fact of deciding up to which level we needed to apply the
open/close principle and how to connect all components to the core system
have been the main difficulties.
Applying the relevant principle indeed requires to write more code than 
what you could if you just wanted to be ``locally efficient'' (with no 
intention to modify the code) and also often requires more data structures
to be created to assure good transition of information between classes.
With an exterior point of view, we can now say that those difficulties
led us to a lot of discoveries in the~\textsc{Java} programming language.
This can be especially be seen in programming where you really learn
by trying to code something, eventually observing something is wrong
and finally making it work.

As the project is divided in two parts, the first being the \emph{core}
and the second the \emph{interface} of the \lstinline|MyFoodora| system,
this report present the guidelines of this part's implementation.
A brief summary of the project \emph{background} and \emph{requirements} will
be detailed in section~\ref{sec:background}.
Next, in section~\ref{sec:analysis_and_design}, will first be presented a general \emph{overview} 
and \emph{analysis} of our solution, second a detailed explanation of how and why
the various design patterns are implemented.
The most interesting parts and functions of our \emph{code} will then be
thoroughly detailed in section~\ref{sec:implementation}.
The different \emph{tests} we used all along the implementation
and the \emph{results} obtained by our solution will respectively
be shown in sections~\ref{sec:testing} and~\ref{sec:results}.
Finally, section~\ref{sec:conclusion} will allow us to draw 
a \emph{conclusion} and consider \emph{further work}.

% section introduction (end)