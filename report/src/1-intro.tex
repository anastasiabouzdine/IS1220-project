\section{Introduction} % (fold)
\label{sec:introduction}
Nowadays, more and more software programs are needed and designed every day.
They play an important role in everyday life and more specifically
in domains like ``\textit{manufacturing, banking, travel, communications,
defense, medicine, research, government, education, entertainement,
law, etc.}''\cite[p.6]{long2008critical}.
The goal of this project is to develop a software solution,
called \lstinline|MyFoodora|, whose functionality is similar to today's food delivery systems\footnote{such as \url{https://www.foodora.fr/}
and \url{https://deliveroo.fr/}.}.

We did not only want to create a program that \emph{works} under the given conditions,
but one that is \emph{resusable} and easily \emph{extendable}.
To meet our expectations, we applied \emph{design patterns} in order
to follow the well-known \emph{open/close principle}.
We thourougly thought and discussed which patterns to follow
and to implement before effectively writing down the code.
This brainstorming approach allowed us
\begin{enumerate}
  \item to keep a global overview of what was needed at which point and why,
  \item to acquire a good comprehension of the advantages and disadvantages of each
  design pattern.
\end{enumerate}

The implementation of the solution has not been done without \emph{difficulty}.
Basically, the main discussion was about the question to which point we needed to apply the
open/close principle and how to connect all components to the core system.
Applying the open/close principle requires indeed to write more code than 
needed if you just wanted to be ``locally efficient'' (with no 
intention to extend the code) and also often requires more data structures to assure a structured and closed transition of information between classes that is coherent everywhere.
Looking back, we can now say with certainty that those difficulties
led us to many discoveries in the~\textsc{Java} programming language.
We did not only learn ``new programming skills", but more importantly we really 
understood why certain methods, we saw in class earlier, are efficent and recommended to implement (using getters and setters f.e.). 

As the project is divided into two parts, the first being the \emph{core}
and the second the \emph{interface} of the \lstinline|MyFoodora| system,
this report presents both the guidelines of the first part's implementation
and of the user interfaces and results, which account for the second part.
A brief summary of the project's \emph{background} and \emph{requirements} will
be detailed in section~\ref{sec:background}.
Next, in section~\ref{sec:analysis_and_design}, we will first present
a general \emph{overview} and \emph{analysis} of our solution, second a detailed 
explanation of how and why the various design patterns are implemented.
The most interesting functions of our \emph{code} will then be
thoroughly detailed in section~\ref{sec:implementation}.
The different \emph{tests} we used along the implementation
and the how precisely we did them will
be shown in section~\ref{sec:testing}.
Linking the core system to a real user is done using what is called
user interfaces, their implementation will be described
in section~\ref{sec:user_interface}.
To be sure everything works and that our system can be
effectively used leads to showing results in section~\ref{sec:results}.
Finally, section~\ref{sec:conclusion} will allow us to draw 
a \emph{conclusion} and consider \emph{further work}.

% section introduction (end)