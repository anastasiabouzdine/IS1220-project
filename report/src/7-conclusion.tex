\section{Conclusion and further work} % (fold)
\label{sec:conclusion}
We have now defined the necessary \emph{background} to understand
what the \emph{requierements} were about and explained how we \emph{designed} 
our solution to meet them. We also described how this 
\emph{implementation} was conducted effectively throughout the code
and what \emph{testing} methods have been used to ensure stability
and correctness.

See who did what in the appendix~\ref{app:who_did_what}.

% advantages/limitations of your solution
% why we did not use test driven development
% what about one test for each class

The reader probably noted the absence of a \emph{results} section.
This is justified by the fact that the scenarios have less importance
since they are already done by \textsc{JUnit} tests. Actual results are more
reprentative as soon as the
user interface will be implemented.
This leads us to the second part of the project which will
consist of implementing a \texttt{command line user interface} and 
an associated \texttt{graphical user interface} to use the \MyFoodora~system.

The way we see the realisation of CLUI and GUI now, is to apply a specific combination of core and user methods for each CLUI command and its arguments.
Once the CLUI is implemented, the next part is about linking
information between actions on the graphical interface to the command line.


% section conclusion (end)