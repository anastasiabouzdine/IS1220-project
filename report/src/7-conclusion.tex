\section{Conclusion and further work} % (fold)
\label{sec:conclusion}
We have now defined the necessary \emph{background} to understand
what the \emph{requierements} were about and explained how we \emph{designed} 
our solution to meet those. We also described how this 
\emph{implementation} was conducted effectively throughout the code
and what \emph{testing} methods have been used to ensure stability
and correctness.

See who did what in appendix~\ref{app:who_did_what}.

% advantages/limitations of your solution
% why we did not use test driven development
% what about one test for each class

The reader probably noted the absence of a \emph{results} section.
This is justified by the fact that those scenarios have less importance
when they are done in a form of \textsc{JUnit} tests and are less
reprentative of which functions will be really called when the
user interface will be implemented.
This leads us to the next second part of the project which will
consist in implementing a \texttt{command line user interface} and 
an associated \texttt{graphical user interface}.

The way we see it now is to apply a specific combination of core
and users methods for each CLUI command and its arguments.
Once the CLUI is implemented, the following part will be to link
information between actions on the graphical interface to
what can be done in the command line.


% section conclusion (end)