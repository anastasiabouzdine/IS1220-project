\subsection{Graphical user interface} % (fold)
\label{sub:graphical_user_interface}

In order to connect the core to a \GUI~we used the \lstinline|java.swing|~package.
In the following, we will describe the general design structure of the gui package including 
its most important features. Then, we will give account of how errors and exceptions produced 
by the user are handled and how we tried to make the application as handy as possible. 
Lastly, we will discuss how we tested the \GUI~and how it should be used.

\subsubsection{General design} % (fold)
\label{ssub:general_design}

The \lstinline|gui|~package includes the following classes: 
\StFra,k\UserF,k\RestF,k\ManagF,k\CourF,k\CustF,k\lstinline|DisplayMealDish|,k\lstinline|LaunchGUI|~and 
\lstinline|StartFrameTest|. 
\lstinline|LaunchGUI| is there only to launch the gui (open the first frame), whereas 
\lstinline|StartFrameTest| takes care of testing some functionality.

We decides to build the \GUI~on two frames: One used for the registration and the loggin of a user
 (implemented via \StFra)
and the one for all functionality a user has as soon as he is logged in (implemented via  
\UserF, \RestF, \ManagF, \CourF, \CustF and \lstinline|DisplayMealDish|). 

To better understand how the frames are connected and how they are put in place by the classes, 
please take a closer look at figure~\ref{fig:start2user_uml} and figure~\ref{fig:userframe2users_uml}.
How exactly does the process work?
We decided to use a design pattern that is similar to the strategy design pattern, where as we use a 
the reference attribute \lstinline|CurrentLogInUser|~declared as a \UserF~to log in to different 
users (see listing~\ref{lst:login_user_GUI}). Thus the abstract class \UserF~forces each of the user 
classes to implement the \lstinline|getInstance()|~function to open up the individualized second frame. 
The advantages are: First, that we have a basic frame 
for all users that can individually be adapted and second, that we have saved the current user frame 
and can easily call it within the start frame. 
Whenever the log in button is succesfully clicked the first frame's visibility is set to 
\lstinline|false|~and the second frame opens. Logically, when the log out button is clicked the first 
frame's visibility is set to \lstinline|true|~and the second frame becomes invisible. 
This pattern allowed us to have a smooth and structured transition between the log in and register 
functionality used by every user and the individual user functionality all deriving from the same 
frame.

Next, we want to take a closer look of how we designed a clear and efficient structure for the 
different user's functionality by using \emph{nested classes}. Since each user can execute 
different functions, we decided to equip both the abstract \UserF~with some basic action classes to 
call setters and getters that are the same for all users as well as the user frames with more 
individualized action classes. These action classes provide action objects that were added to the 
respective menu bars of each user class frame and will execute their \textit{action} when clicked on. 
To can a better feeling of what we described in this paragraph, please take a look at the 
figure~\ref{fig:restaurantNestedClass_uml} and the listing~\ref{lst:nested_class_rest} both of which 
are using the \RestF~as an exemple. 
Every user frame has the basic menu items: Settings and Information as well as possible other 
individual ones. Clicking on a functional menu button will therefore trigger the functionality and 
execute the funtion. This design firstly gives a very clear overview of which functionality can be 
added to which menu bar item and is a very efficient way to handle many, but easy functionalities in 
a frame which is the case for our users (displaying and setting personal date as well as adding and 
removing dishes f.e.).

% subsubsection handling_of_the_gui (end)
\subsubsection{Handiness and Error handling} % (fold)
\label{ssub:handling_of_the_gui}

We tried to make the \GUI~as user friendly as possible by using a consistent and structured design 
and giving instructions when needed as partly already described above in 
section~\ref{ssub:general_design}. For that we will use buttons for the navigation in \StFra~as it 
has only two basic functionalities: log in and register. In the different user frames we also use 
buttons for basic navigation and the menu bar to execute the respective functionalities. 
For each menu bar action we added a little discription appearing when the mouse is held over the item 
as can be seen in the figure~\ref{fig:menuBarDescription}. Additionally we made sure that complicated 
functions like the ordering process of the client is explained in a pop up window (see figure~\ref{fig:instructionPopUp})
The handiness should be checked out by the user himself, but will also briefly demonstrated in a 
short video.

In the second part of this section, we quickly want to cover how we decides to handle bad user 
entries. When ever an exception is thrown due to a bad entry by the user, we catch it and will inform 
the user with the help of a pop up window. To get a better feeling for that, please take a look at 
figure~\ref{fig:wrongFormat}. We tried to make sure that the instructions are precise (giving clear 
instructions) and adapted for the different exceptions that can pop up.

% subsubsection general_design (end)
\subsubsection{Testing and Usage} % (fold)
\label{ssub:testing_and_usage}

Finally, we will explain how we tested and why we tested this way. 
When starting the \GUI~we weren't sure how we could test since all the correctness of the different 
functions was already tested in the first part of the project. Since it is not possible to test 
whether clicking on a button makes appear a certain frame or panel with \lstinline|JUnit|~tests, we 
found a way to implement so called \emph{bots} that will take control over mouse and be programmed to 
move on a certain position on the screen and click a button there. We implemented some tests for the 
\StFra~in the beginning to log in and register users, but we quickly realised that these tests will 
not be consistent if the layout of the frame changes (f.e. a button moving to a different position) 
and that there are to many possibilities to test. That is why, we decided to do most of the testing 
by hand afterwards. All the tests we implemented in the beginning can be found in 
\lstinline|StartFrameTest|~and you are invited to use them.

% subsubsection testing_and_usage

% subsection graphical_interface (end)

