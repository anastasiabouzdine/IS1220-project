\subsection{What differs from requirements ?} % (fold)
\label{sub:what_differs_from_requirements}

\paragraph{Meal creation in commandline} % (fold)
\label{par:meal_creation_in_commandline}
The first command line command which has been subject to a change
was the \lstinline|createMeal| command. It's new form is
\begin{center}
  \lstinline|createMeal <mealName> <mealType>|
\end{center}
this change allows the restaurant to specify immediately which type of meal
he wants to create, it can either be a fullmeal or a halfmeal.
This prevents the system from failing in case the \lstinline|addDish2Meal|
is never called after this one.
The restaurant will immediately know which type of dishes he has to add
to the meal in order to stay relevant with the mealtype he gave. 
% paragraph meal_creation_in_commandline (end)

\paragraph{Order creation in commandline} % (fold)
\label{par:order_creation_in_commandline}
Then, we changed the \lstinline|createOrder| command. It's new form is
\begin{center}
  \lstinline|createOrder <restaurantName>|
\end{center}
This is justified by the fact that naming an order by the customer
has is of no use in our project nor in reality as the system
handles the assignement of a unique ID of the order.
One can then respond that having an ordername allows to handle
multiple orders at the same time by the same customer,
which is indeed true but of no use either.
An order isn't restricted to a certain amount of dishes
or meals and thus a logged in customer will only
have to handle (create, add items to it and then end it)
the \emph{current order}.
% paragraph order_creation_in_commandline (end)

\paragraph{Add an item to order with a quantity} % (fold)
\label{par:add_an_item_to_order_with_a_quantity}
A new method called \lstinline|addNbItem2Order| has been implemented
in case the customer wants to order the same item more than once
and doesn't want to rewrite the same command multiple times.
This case scenario is useful in reality for group orders.
% paragraph add_an_item_to_order_with_a_quantity (end)

\paragraph{No need to specify the username for \texttt{onDuty}
and \texttt{offDuty} methods} % (fold)
\label{par:no_need_to_specify_the_username_for_lstinline_onduty_and_lstinline_offduty_methods}
As this function is used by the currently logged in courier,
our implementation of the core allows to keep track of the logged in user
and the latter doesn't need to specify his own username when changing his status.
% paragraph no_need_to_specify_the_username_for_lstinline_onduty_and_lstinline_offduty_methods (end)

\paragraph{Use of \texttt{showMenuItem} by any type of user} % (fold)
\label{par:use_of_lstinline_showmenuitem_by_any_type_of_user}

% paragraph use_of_lstinline_showmenuitem_by_any_type_of_user (end)

% subsection what_differs_from_requirements (end)